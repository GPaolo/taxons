Fastsim is a fast, lightweight simulator of a wheeled robot (khepera-\/like).

This is the library version. A Sferes2 module and ros module are available in separate repositories. The R\+OS module currently uses this library, but not the sferes one (not yet).

{\itshape If you use this software in an academic article, please cite\+:}

Mouret, J.-\/B. and Doncieux, S. (2012). Encouraging Behavioral Diversity in Evolutionary Robotics\+: an Empirical Study. Evolutionary Computation. Vol 20 No 1 Pages 91-\/133.

\subsection*{Usage \& installation }


\begin{DoxyItemize}
\item run {\ttfamily ./waf configure} and {\ttfamily ./waf build}
\item running\+: {\ttfamily build/src/test\+\_\+fastsim worlds/example.\+xml}
\end{DoxyItemize}

\subsubsection*{Depedencies\+:}


\begin{DoxyItemize}
\item S\+DL 1.\+2 (can be deactivated)
\end{DoxyItemize}

\subsection*{Academic papers that use faststim }


\begin{DoxyItemize}
\item Jingyu Li, Jed Storie, and Jeff Clune. (2014). Encouraging creative thinking in robots improves their ability to solve challenging problems. In Proceedings of the 2014 conference on Genetic and evolutionary computation (G\+E\+C\+CO \textquotesingle{}14). A\+CM, New York, NY, U\+SA, 193-\/200.
\item Doncieux, S. (2014). Knowledge Extraction from Learning Traces in Continuous Domains. A\+A\+AI 2014 fall Symposium \textquotesingle{}\textquotesingle{}Knowledge, Skill, and Behavior Transfer in Autonomous Robots\textquotesingle{}\textquotesingle{}. Pages 1-\/8.
\item Koos, S. and Mouret, J.-\/B. and Doncieux, S. (2013). The Transferability Approach\+: Crossing the Reality Gap in Evolutionary Robotics. I\+E\+EE Transactions on Evolutionary Computation. Vol 17 No 1 Pages 122 -\/ 145
\item Doncieux, S. (2013). Transfer Learning for Direct Policy Search\+: A Reward Shaping Approach. Proceedings of I\+C\+D\+L-\/\+Epi\+Rob conference. Pages 1-\/6.
\item Doncieux, S. and Mouret, J.\+B. (2013). Behavioral Diversity with Multiple Behavioral Distances. Proc. of I\+E\+EE Congress on Evolutionary Computation, 2013 (C\+EC 2013). Pages 1-\/8.
\item Mouret, J.-\/B. and Doncieux, S. (2012). Encouraging Behavioral Diversity in Evolutionary Robotics\+: an Empirical Study. Evolutionary Computation. Vol 20 No 1 Pages 91-\/133.
\item Mouret, J.-\/B. (2011). Novelty-\/based Multiobjectivization. New Horizons in Evolutionary Robotics\+: Extended Contributions from the 2009 Evo\+De\+Rob Workshop, Springer, publisher. Pages 139--154.
\item Pinville, T. and Koos, S. and Mouret, J-\/B. and Doncieux, S. (2011). How to Promote Generalisation in Evolutionary Robotics\+: the Pro\+G\+Ab Approach. G\+E\+C\+CO\textquotesingle{}11\+: Proceedings of the 13th annual conference on Genetic and evolutionary computation A\+CM, publisher . Pages 259--266.
\item Koos, S. and Mouret, J.-\/B. and Doncieux, S. (2010). Crossing the Reality Gap in Evolutionary Robotics by Promoting Transferable Controllers. G\+E\+C\+CO\textquotesingle{}10\+: Proceedings of the 12th annual conference on Genetic and evolutionary computation A\+CM, publisher . Pages 119--126.
\item Doncieux, S. and Mouret, J.-\/B. (2010). Behavioral diversity measures for Evolutionary Robotics. W\+C\+CI 2010 I\+E\+EE World Congress on Computational Intelligence, Congress on Evolutionary Computation (C\+EC). Pages 1303--1310.
\item Mouret, J.-\/B. and Doncieux, S. (2009). Overcoming the bootstrap problem in evolutionary robotics using behavioral diversity. I\+E\+EE Congress on Evolutionary Computation, 2009 (C\+EC 2009). Pages 1161 -\/ 1168.
\end{DoxyItemize}

\subsection*{Minimal documentation }

Fastsim uses a bitmap (a pbm file) as an environment. It uses pixel-\/wise collision detection and interesection tests (lasers). There are two coordinate systems\+: world coordinates, and pixel coordinates. In general, everything is expressed in world coordinate and a scaling is automatically applied by fastsim to get the pixel coordinates.

Units\+:
\begin{DoxyItemize}
\item everything is expressed in world coordinates
\item angles are in radians {\itshape except in the X\+ML file}, where they are in degrees.
\end{DoxyItemize}

Main classes\+:
\begin{DoxyItemize}
\item {\ttfamily Map(const char$\ast$ fname, float real\+\_\+w)}\+: the main object of fastsim. It contains the robot, the goals, and the illuminated switches. {\ttfamily real\+\_\+w} corresponds to the width of the map in world coordinates. Do not forget to call Map\+::update() at each time step
\item {\ttfamily Robot(float radius)}\+: a khepera-\/like robot with a differential drive system. You need to add sensors to customize it (lasers, camera, radars, etc.). Bumper are added automatically.
\item {\ttfamily Display(const boost\+::shared\+\_\+ptr$<$Map$>$\& m, const Robot\& r)}\+: a S\+DL buffer (window) that can be used to visualize the robot, the environment, etc. (depends on S\+DL)
\item {\ttfamily Settings(const std\+::string\& xml\+\_\+file)}\+: a simple X\+ML parser to easily implement a configuration file for fastsim (see below)
\end{DoxyItemize}

Objects\+:
\begin{DoxyItemize}
\item {\ttfamily Goal(float x, float y, float diam, int color)}\+: goals are omnidirectional beacons that can be seen by {\ttfamily Radar} sensors. Depending on the configuration of the radar, the robot can see or not see goals through walls. The activated slice is -\/1 if the goal is not visible.
\item {\ttfamily Illuminated\+Switch(int color, float radius, float x, float y, bool on)}\+: illuminated switches are omnidirectional beacons that can be hidden by walls. They can be switched on or off ({\ttfamily activate()}/{\ttfamily deactivate()}). An {\ttfamily Illuminated\+Switch} can be linked to other switches ({\ttfamily link()}) to create \char`\"{}circuits\char`\"{}. Once put in a map, the illuminated switches are switched on if a robot touches it.
\end{DoxyItemize}

Sensors\+:
\begin{DoxyItemize}
\item {\ttfamily Laser(float angle, float range, float gap\+\_\+dist = 0.\+0f, float gap\+\_\+angle = 0.\+0f)}\+: a laser telemeter. The direction (relative to the robot frame) is controlled by the {\ttfamily angle}. If the robot does not see anything (no object in the range), then the distance is -\/1
\item {\ttfamily Light\+Sensor(int color, float angle, float range)}\+: A Light\+Sensor senses an Illuminated\+Switch of the same color. The range is an angular range (there is no distance range), the angle is the orientation of the sensor relative to the robot. If there are several Illuminated\+Switches of the same color, the exact behavior is undefined.
\item {\ttfamily Radar(int color, int nb\+\_\+slices, bool through\+\_\+walls = true)}\+: A radar sensor senses a Goal object of the same color. It is a circular sensor divided into slices. If a slices \char`\"{}see\char`\"{} a Goal of the right color, then this slice is activated. There can be only one slice activated at a time. If there are seveval goals of the same color, the behavior of the radar is undefined.
\end{DoxyItemize}

\subsection*{Example with a configuration file }


\begin{DoxyCode}
1 \{C++\}
2 
3 #include <iostream>
4 #include "fastsim.hpp"
5 
6 int main(int argc, char** argv) \{
7   using namespace fastsim;
8   assert(argc == 2);
9   fastsim::Settings settings(argv[1]);
10   boost::shared\_ptr<Map> map = settings.map();
11   boost::shared\_ptr<Robot> robot = settings.robot();
12 
13   Display d(map, *robot);
14 
15   for (int i = 0; i < 10000; ++i)
16     \{
17       d.update();
18       robot->move(1.0, 1.1, map);
19       usleep(1000);
20     \}
21   return 0;
22 \}
\end{DoxyCode}


Configuration file\+: 
\begin{DoxyCode}
1 <?\textcolor{keyword}{xml} \textcolor{keyword}{version}=\textcolor{stringliteral}{"1.0"}?>
2 <\textcolor{keywordtype}{fastsim}>
3   <\textcolor{keywordtype}{display} \textcolor{keyword}{enable}=\textcolor{stringliteral}{"true"}/>
4   
5   <\textcolor{keywordtype}{map} \textcolor{keyword}{name}=\textcolor{stringliteral}{"worlds/cuisine.pbm"} \textcolor{keyword}{size}=\textcolor{stringliteral}{"600"}/>
6   <\textcolor{keywordtype}{robot} \textcolor{keyword}{x}=\textcolor{stringliteral}{"300"} \textcolor{keyword}{y}=\textcolor{stringliteral}{"300"} \textcolor{keyword}{theta}=\textcolor{stringliteral}{"0"} \textcolor{keyword}{diameter}=\textcolor{stringliteral}{"30"}/>
7   <\textcolor{keywordtype}{illuminated\_switch} \textcolor{keyword}{x}=\textcolor{stringliteral}{"250"} \textcolor{keyword}{y}=\textcolor{stringliteral}{"450"} \textcolor{keyword}{color}=\textcolor{stringliteral}{"0"} \textcolor{keyword}{radius}=\textcolor{stringliteral}{"10"} \textcolor{keyword}{on}=\textcolor{stringliteral}{"true"}/>
8   <\textcolor{keywordtype}{light\_sensor} \textcolor{keyword}{angle}=\textcolor{stringliteral}{"100"} \textcolor{keyword}{color}=\textcolor{stringliteral}{"0"} \textcolor{keyword}{angular\_range}=\textcolor{stringliteral}{"50"}/>
9   <\textcolor{keywordtype}{goal} \textcolor{keyword}{x}=\textcolor{stringliteral}{"100"} \textcolor{keyword}{y}=\textcolor{stringliteral}{"100"} \textcolor{keyword}{color}=\textcolor{stringliteral}{"0"} \textcolor{keyword}{diameter}=\textcolor{stringliteral}{"10"}/>
10 
11   
12   <\textcolor{keywordtype}{laser} \textcolor{keyword}{range}=\textcolor{stringliteral}{"100"} \textcolor{keyword}{angle}=\textcolor{stringliteral}{"45"}/>
13   <\textcolor{keywordtype}{laser} \textcolor{keyword}{range}=\textcolor{stringliteral}{"100"} \textcolor{keyword}{angle}=\textcolor{stringliteral}{"-45"}/>
14   
15   <\textcolor{keywordtype}{radar} \textcolor{keyword}{slices}=\textcolor{stringliteral}{"4"} \textcolor{keyword}{color}=\textcolor{stringliteral}{"0"}/>
16 </\textcolor{keywordtype}{fastsim}>
\end{DoxyCode}


\subsection*{Example without a configuration file }


\begin{DoxyCode}
1 \{C++\}
2 #include <iostream>
3 #include "fastsim.hpp"
4 
5 int main()
6 \{
7   try
8     \{
9       using namespace fastsim; 
10       boost::shared\_ptr<Map> m = boost::shared\_ptr<Map>(new Map("cuisine.pbm", 600));
11       m->add\_goal(Goal(100, 100, 10, 0));
12       Robot r(20.0f, Posture(200, 200, 0));
13       r.add\_laser(Laser(M\_PI / 4.0, 100.0f));
14       r.add\_laser(Laser(-M\_PI / 4.0, 100.0f));
15       r.add\_laser(Laser(0.0f, 100.0f));
16       r.add\_radar(Radar(0, 4));
17       Display d(m, r);
18 
19       float x = 30;
20       for (int i = 0; i < 10000; ++i)
21   \{
22     d.update();
23     r.move(1.0, 1.1, m);
24   \}
25     \}
26   catch (fastsim::Exception e)
27     \{
28       std::cerr<<"error : "<<e.get\_msg()<<std::endl;
29     \}
30   return 0;
31 \}
\end{DoxyCode}
 